% !TeX spellcheck = en_GB

\documentclass[12pt,a4paper]{scrartcl}
\usepackage[margin=2.5cm, bottom = 3cm, right = 2.5cm]{geometry} 
\usepackage[]{ragged2e} %text full justificaton
\usepackage[onehalfspacing]{setspace} %\onehalfspacing %\setstretch{1,6}
%\onehalfspacing
\usepackage{tabularx}
\usepackage{paralist}
\setlength{\plitemsep}{1ex} 
\setlength{\pltopsep}{1ex}
%\usepackage{hyperref}

\setlength{\parindent}{0pt} %no paragraph indent
\setlength{\parskip}{2.5ex plus0.5ex minus0ex} %skip between pargagraps
%\renewcommand{\baselinestretch}{1.5} 
%\setlength{\baselineskip}{1.25}
%\linespread{1.25} %1.0 - single, 1.3 - ohe and half, 1.6 - double 

\usepackage[main=english, ngerman]{babel} %language used document structure, e.g. citations
 
%works only with XeLaTeX - compiler
\usepackage{fontspec}
\setromanfont{Times New Roman} %Calibri
\setsansfont{Arial}
%\setmonofont[Color={0019D4}]{Courier New}


\usepackage{amsmath, amsfonts, amssymb}
\usepackage{graphicx}

\usepackage{multicol}
\setlength{\columnsep}{1cm}

\usepackage{comment}

\usepackage[backgroundcolor = orange!50, bordercolor = orange!50]{todonotes}

%type of citation, pick up your own
\usepackage[natbibapa]{apacite} %\usepackage{natbib}
\bibliographystyle{apacite} %plain, apalike, aparefs (apa6 class only), IEEEtran



\title{Contribution of Wassily Leontief to The History of Economic Thought in the 20th Century}
\author{Bystrov Oleg}
\date{\today}



\begin{document}
	%makes title 
	\maketitle
	

History of Economic Thought in the 20th Century \newline
Dr. Annette Vogt \newline
\newline
Bystrov Oleg \newline
MatrikelNr: 567843


	
	\clearpage

	
	%you can write right here, but it is easier to manage big projects if you have separate file for each section
%	

\begin{itemize}
	\item list with bullet points
\end{itemize}

\begin{enumerate}
	\item a
\end{enumerate}


	
\section{Introduction} \label{introduction}


Known for his contribution of the Input-Output Analysis, Wassily Leontief (1906 - 1999) has been awarded the Nobel Prize in 1973 honoring "the development of input-output method and for its application to important economic problems." Input-Output Analysis can be used for calculating interdependence of components within an economic production system of any size, from a private company to a whole country. This method has found wide usage across the globe in a multitude of economic fields, both in the western and soviet worlds.

In Section 2 we look at Leontief from a personal perspective, his educational background and his political tendencies This will then enable us to put his work in context. The notion of looking at economic accounts as a set of interconnected and interdependent pieces was not a completely new concept when Leontief developed his method, though he certainly improved upon earlier attempts and made them actually useful and actionable. \cite[p.~512]{Baumol2009} \todo{confirm the statement}. In this paper we will look at the influences that lead Leontief to introduce his work to the world in \ref{influences} and improvements along the road in section \ref{analysis}.\todo{might remove this}

The Nobel Memorial Prize in Economic Sciences is awarded to "those who [...] shall have conferred the greatest benefit on mankind." \cite[]{Nobel} in the field of economics. Some 37 years of refinements and expanding usage passed from the time Leontief constructed his first input-output tables in 1936 when he calculated the figures based on actual historical data for "the Economic System of the United States, 1919" \cite[]{kohli2003leontief}. Since then this method has been acclaimed across many countries and industries to improve planning and understand relations withing an economic system. Leontief supported the usage and improvement of the method throughout his career as well as contributed strongly in his effort to "making quantitative data more accessible and more indispensable to economics" \cite[p.75]{dymond2015recent} , thus transforming the nature of economics in general to be more of a formal science rather than a social one.

	
	%\begin{multicols}{2}
	[
	\section{First Section}
	All human things are subject to decay. And when fate summons, Monarchs must obey.
	]
	Hello, here is some text without a meaning.  This text should show what 
	a printed text will look like at this place.
	If you read this text, you will get no information.  Really?  Is there 
	no information?  Is there...	
\end{multicols}
	
	\section{Life}
	
	Leontief was born in the year 1905 in Munich, Germany. His father, at the time, was defending his doctoral dissertation and met his mother there, later he became a professor of economics in St. Petersburg University. \cite[pp.347-355]{Kaliadina2006} \cite[]{wassily.leontief.server}
	
	\subsection{Early Life and Education}
	
	Leontief has been home-schooled and attended school only for the last two years to get the attestation. For this reason Leontief managed to enter the university at the earlier age of fifteen in the year 1921. He joined the socio-economic department in the Faculty of Social Sciences at the St. Petersburg University. 
	
	In the interview conducted in 1990 \cite[]{Kaliadina2006} Leontief explains his decision to emigrate. At that time the soviet government was in the process of dissolving free speech and suppressing the "intelligentsia" of the country. Leontief has been involved in debates, criticizing the regime, at his University and was taken multiple times into custody by the Cheka - the first of an upcoming array of security organisations of the Soviet State. There he has been questioned and released, going back to the university and subsequently starting the process again. Leontief, as he stated himself, considered his work as a scientist to be the most important thing in his life and "understood that it was not possible to work as a scientist" \cite[p.351]{Kaliadina2006} in the Soviet State. But leaving the country was not a simple matter. Leontief was presumed to have a terminal illness and was allowed to be issued a passport. Later it turned out to be false. After graduating from the University of Leningrad in 1925 Leontief left the Soviet Union.
	
	It is worth noting that in 1922, after protests of the teaching staff in response to planned spending cuts on higher education, it was decided by the government to exile many of them from the country. This included professor P.A. Sorokin, who taught sociology at the Petrograd University. Sociology was not a mandatory subject for Leontief, yet he was interested in the topic and took a class of professor Sorokin. Later they would meet again, both teaching at Harvard in the United States. \cite[p.353]{Kaliadina2006}
	
	After leaving the Soviet territory in 1925 Leontief first came to Berlin where he got his PhD in 1928 with the Thesis topic being: "The Economy as a circular flow. Structural Change and Economic Dynamics". It laid the theoretical groundwork for what would soon manifest itself as Input-Output Analysis.
	
	\subsection{Work Outline}
	
	After achieving the PhD Leontief went to work for one year to the Institute for World Economics in 1928 (here the idea of the input-output approach was developed).
	
	Between 1929 and 1931 he interrupted his work and spent a year in China, where he was an advisor to the Ministry of Railways.
	
	In 1931 he received an appointment to the National Bureau of Economic Research in New York.
	
	In 1932 he moved to Harvard and stayed there as first as an instructor and later as a professor until 1975.
	
	From 1975 he has taught at the New York University almost until his death in 1999.
		
	\subsection{Political Stances}
	
	Leontief saw the role of an economist as apolitical and rather a scientific one. He was interested in understanding the system and putting it in a mathematical form and empirical context, rather than influencing politics. At certain points he was commissioned to advise government institutions, mostly applying economic methods to calculate the best course of action for them, by understanding the industry and interindustry relationships and applying his technical knowledge, but never entered the game of altering economic policy. \cite[p.21]{Hamilton2008}
	
	\section{Influences} \label{influences}
	
	To understand how this method came to be we should look at the knowledge that Leontief had accumulated during his academic career and the work of economists who have had an influence. 
	
	The idea to describe an economy as a set of interindustry activities (co-related and co-dependent) came before Leontief. Evidence of this approach can be found going back to the seventeenth century when Sir William Petty, described as the first econometrician, presented the notion of a "circular flow" in 1690. Here he states that assessing the wealth of a nation lies in reflecting the interrelationships between production, distribution and disposal of value within its economy. The first estimates of national economic accounts appeared soon after.
	
	At that time Mercantilism was the common economic strategy. It emphasized the need of exports to grow the wealth of an economy and, thus, pushed for higher exports and lower imports by adjusting economic policy to its advantage. During the eighteenth century another school of economic thinkers rose, which was called the Physiocracy. They believed that the wealth of a nation amounted solely to their "produit net" - meaning the agricultural product. This opposed the popular Mercantilist view at the time and caused Physiocracy to be considered as controversial. François Quesnay was the leading Physiocrat. In 1759 he developed his Tableau Économique \todo{add Tableau as image}. The Tableau described how value flows from one participant to another, and apart from the notion that manufacturing and commerce did not contribute value to the economy (they were called "sterile expenditures") has picked up a correct thinking about how economy functions. Coming from a Physiocrat the Tableau has been widely criticized by proponents of opposing political views and has not receive wide attention of the economists at the time. Yet it has received praise from some of the economists. Most notably in our context, one of them was Marx. \cite[pp.724-732]{Miller2009}
	
	Marx had been developing his own theory and has found that Quesnay's Table supports his own ideas of surplus value (which grew into the "Labor Theory of Value" described in "Capital" in 1867 \cite[]{Marx1887}). Praising it as a "brilliant idea" in a letter to Engels \cite[p.144]{Baumol2000} and in his work Marx once again drew attention to the Tableau.
	
	In the early 1920s, during his undergraduate studies, Leontief learned Marx. Nevertheless, as he states himself, his work has not been influenced by Marx's schemes of reproduction. While a student he "read systematically all economists" \cite[p. 53]{Hamilton2008}. In fact, Leontief developed his analysis in response to shortcomings of the classical–neoclassical supply-and-demand analysis as well as lack of facts in the general equilibrium theory.
	
	Ultimately Marx, having revived the Tableau of Quesnay, has failed to provide the statistical framework to make it operative and useful in a practical context. \cite[p.729]{Miller2009} \cite[p.232]{Schumpeter1954}
	
	It is not realistic to pinpoint exactly, yet it is likely that it was through Marx that he had been introduced more closely to the work of Quesnay. During his career Leontief has pushed efforts to describe concepts in a systematic way, put value on empirical verification, as well as being a strong proponent of introducing hard sciences within more vague social science of economics, which it was extensively so at the time. The Tableau posed a potential for real-life applications. \cite[p.16, p.20]{Hamilton2008}
	
	In 1925, when Leontief joined the University of Berlin he became a research assistant of Prof. Werner Sombart who was a historical economist. Later when Leontief started work on his doctorate Sombart was his primary thesis advisor. Sombart could not follow the Leontief's mathematics and Ladislaus von Bortkiewicz, a mathematical economist, joined as second advisor \cite[p.513]{Baumol2009}. He was one of critics of Marx (which resonated with Leontief's own point of view). Most importantly he helped express Leontief's framework in an algebraic form.
	
	Bortkiewicz has been one of the central figures in shaping Leontief's work, despite the appearance that Leontief has been a man of his own accord, and of his own opinion on things. Bortkiewicz writes to the Dean of the University of Berlin in a confidential appraisal of Leontief's PhD application: "In developing his - in my opinion very doubtful - theoretical constructs the candidate received no guidance whatsoever from his academic teachers. He arrived at his present position quite independently, one might say, despite them. It is very likely that he will maintain this scientific point of view also in the future." \cite[p.179]{Samuelson1991}
	
	What Quesnay's Tableau was lacking is the mathematical formalization and empirical validation. With "The Economy as a circular flow" \cite[]{Leontief1991} Leontief introduced a method foundation this will prove itself transformational for the possibilities of practical economic planning during the coming decades.
	
%	"acquired new actuality in our time through the great work of Leontief,11 which, entirely different though it is from Quesnay’s in purpose and technique, nevertheless revived the fundamental principle of the tableau method. Marx, who stands between the two, did not attempt to make his schema statistically operative."
	
%	"Economics, like every other science, started with the investigation of ‘local’	relations between two or more economic quantities, such as the relation between the price of a commodity and the quantity of it that is available in a market"
	
%	"It was but slowly that the fact began to dawn upon analysts that there is a pervading interdependence between all economic phenomena, that they all hang together somehow"
	
%	"they never bothered to investigate how things hang together"
%	\cite[p.232]{Schumpeter1954}
		
%	This one provides insight into other work on calculating economic systems alongside Leontiefs \cite{Bjerkholt2006}
	
	\section{Input Output Analysis} \label{analysis}

	Input-Output Analysis ("I-O") is a method of economic analysis aimed at reflecting interdependencies between different economic accounts. It is used to estimate the impact of certain changes to the economy as a whole and analyzing the ripple effects caused by that change. The basis for the analysis is the input-output table. The table has one row and one column for each sector of the economy. For each pair of sectors (a row and a column) it is calculated the amount of value that flowed directly between them during a stated period.
	
	Work on the Input Output Analysis can be split in two phases. The descriptive phase and the analytical phase. The descriptive phase involves building the tables in accordance to the inputs and outputs of the industries or sectors.
	
	The analytical phase consists of two "piers" . The first pier is building a set of equations describing the output for each industry to all other industries. Basically the sum of all entries in a row. The second pier is another set of equations that describe the amount of input required by an industry from other industries to produce its own output. Thus, illustrating the interdependence between the industries.
	
	The most amount of effort and investment during the preparation of the I-O Analysis is required for the descriptive phase. For a better outcome of the analysis the sectors should have a homogeneous output which means each sector covers a narrow field of an economy, consequently an describing an economy of a country would result in numerous sectors. For each sector there needs to be a value-based estimation for the output of the sector to each other sector. It is not always straightforward to estimate the value of the output for example when dealing with the service related sectors.
	
	The results of Input-Output analysis allows to build formal relationships between parts of an economy and further to explain and predict effects within the system. \cite[]{InvestopediaIO} \cite[pp.137-138]{christ1955review}
	
	\textbf{Possible Applications}
	
	The application areas of such analysis are multiple. From planning production, to understanding problems of economic imbalance which have occurred in the past, to understanding and preparing for an expected shift somewhere in the economy. Its use can be both strategic and operational (assuming the appropriate tables have been developed beforehand).
	
	This information can be used by governments who would like to adjust parts of the economy, as well as by the private sector (most useful to big corporations operating on a large scale) to uncover scarce areas and predict or understand past, current or future demand of their products.
	
	\textbf{Critique and the history of refinements}
	
	Good source of analysis of IO \cite{christ1955review}
	
	\section{Use of the I-O Analysis and its Influence on Economic Thought}  \label{contribution}
	
	Leontief was a strong believer that the role of an economist as a scientist is to understand the system, describe it in a coherent, verifiable theory and to build a framework to model the system in a useful format that is aimed at providing useful application of such framework. Ultimately it should lead to improvements for the people \todo{improve wording and put in line with the other paragraphs}
	
%		As an economist Leontief was not interested in improving the system, but rather understanding how it works. “First job of an economist – understand the economic system”
	
%	Register the facts in a systematic way, to make it possible to explain the operation of a system
	
	Many research projects in economic science are either too theoretical without any practical applicability or increasingly too practical, lacking a solid underlying theory to justify the correctness of such research. This was one of the critiques of the science of economics that Leontief consistently emphasized throughout his career and mentioned in his presidential address delivered in 1970 titled "Theoretical Assumptions and Nonobserved Facts". \cite[]{Leontief1971}
	
	The success and wide adoption of the Input Output Analysis can be attributed to its empirical applicability and practical usefulness. The continued improvement of the method, supported by Leontief himself and technological advancement allowed for it to be implemented in a variety of cases and expanded from reflecting a smaller targeted subset of economic activity to modelling comprehensive and wide representations of an economy of a country.	
	
	Another advantage of input-output analysis is the fact that is it agnostic to the underlying political system. It helped the method to be widely adopted internationally in countries including United Kingdom, Norway, Denmark, the Netherlands, Italy, Canada, and Japan. Even the Soviet Union, which normally shunned any and all western inventions for ideological reasons at some point recognized the value of Input Output and the advantages it provides for planning (the Soviet Union's infamous 5-year-plan) and adjusted it to their ideology. Arguably it also helped that Leontief was a descendant of Soviet education and parts of the method's theoretical foundation could be related and traced back to the work of Karl Marx. More about IO and USSR in \ref{ussr} \cite[p.297]{rose1995} \todo{check this for correctness in the Hamburger Jahrbuch}
	
%	\begin{itemize}
%		\item Positive effects to understanding an economic system caused by IO
%		\item Problem: IO is an expensive undertaking for more complex systems
%	\end{itemize}

% usages in the first pages of https://www.iioa.org/conferences/13th/files/Fontela_Leontief.pdf

	Conducting IO Analysis has one serious implication. One which hindered it's adoption and raised questions about its viability. It is the implication of cost. Producing accurate and therefore valuable IO Tables requires an immense amount of work, which to a high degree can be attributed to the identification and estimation of outputs of the sectors. As such this work could only be facilitated by entities with considerable financial capabilities. These

	\subsection{Government Sector}
	
	\begin{itemize}
		\item Applications in the military (calculation of important production sectors and enemy targeting help)
		\item Introduction in the USSR: \cite{leontief1960niedergang}
	\end{itemize}
	
	\subsection{Private Sector}
	
	Applications private sector 
	
	\subsection{Introduction in the USSR} \label{ussr}
	"Input-output planning was never adopted because the material balance system had become entrenched in the Soviet economy, and input-output planning was shunned for ideological reasons" 
	\todo{todo} https://en.wikipedia.org/wiki/Input–output\_model
	
	Cold-War - USA USSR as 'teachers' of economic policies.\newline
	
	\section{The Award of the Nobel Prize} \label{nobelprize}
	
	18 October 1973 the Prize in Economic Science in Memory of Alfred Nobel went to "the father of Input-Output Analysis" - Wassily Leontief - "for the development of the input-output method and for its application to important economic problems." \cite[]{nobelpress} It is the 5th time the Sveriges Riksbank Prize in Economic Sciences in Memory of Alfred Nobel has been awarded since its institution in 1969
	
	\textbf{The Nobel Prize Speech}
	
	It is customary for a scientist to hold a lecture upon receiving the Nobel Prize. Leontief, in his Nobel Prize Lecture titled "Structure of the world economy: Outline of a simple Input-Output Formulation", makes an introduction to the Input-Output Analysis with a simple example of one country's economy, outlining how the interdependence of the sectors can be observed within the framework of a two-way input-output table. Moreover the flexibility of the table allows processes or entities to fall outside of the analysis. As the scope expands new rows and columns can be added to the table as some of the external inflows and outflows become internalized. This permits for a more detailed description of economic activities, by adding new rows and columns.
	
	At the time of the speech there were efforts underway by the United Nations to construct a database for a systematic input-output study not of a single economy of a country but of the world economy viewed as a system of interrelated parts. The goals of the study were:
	
	\begin{itemize}
		\item  to study results that prospective environmental issues and policies would probably have for world development in the absence of changes in national and international development policies.
		\item to study the effects of alternative policies to promote development and at the same time preserve and improve the environment.
	\end{itemize}

	Ultimately it should serve as help to the world community to make decisions regarding future development and environmental policies as rational as possible.
	
	I-O Analysis hasn't been developed to portray such a complex disintegrated system, but the flexibility of the method allowed it to be adapted to such a use-case. In the lecture Leontief presented a formulation of the method which would provide general framework for collection and organization of factual data needed to describe the world economy.
	
	In the framework the world economies are divided into \textit{Developed} and \textit{Less Developed} regions. Each region consisting of 3 productive sectors: Extraction Industry (produces raw materials), All Other Production (supplying goods and services) and Pollution Abatement Industry. For each regions there is also a consumption sector.
	
	The frameworks presents a range of basic assumptions towards the progression of the world economic trend which allow the analysis to take place. At that point there has not yet been enough data collected for a detailed analysis. Despite that Leontief assumed his usual position of the importance of applying theory into practice. Presented some preliminary results of the framework "involving" fictitious input data, albeit the general order of magnitude were to "reflect crude, preliminary estimates of inter-sectoral flows within and between the Developed and Less Developed regions" \cite[p. 389]{leontief1974} between 1960 and 1970.
	
	The substance of the analysis presented in the lecture consists of a projection of economic changes from the year 1970 to 2000 in the world economies in the wake of environmental concerns. 
	
	As a closing remark Leontief refrained from drawing any conclusions at that point and pointed out the danger that "theories tend to shape the facts they try to explain" (p. 399), noting that their proposed formulation is designed to alleviate this danger. It is required to ascertain the necessary facts before any conclusions can be drawn. \cite[pp. 387-401]{leontief1974}
	
	\textbf{Remarks on the Nobel Prize Speech}
	
		
%	results of the outcome of the UN Study https://www.iioa.org/conferences/13th/files/Fontela_Leontief.pdf

	\section{Conclusion}
	
	It is clear that IO Analysis is flexible enough to adjust its structure to an evolving set of requirements of a particular scientific research. It is further notable that this framework gives way to substitute certain assumptions like general equilibrium, fixed/fluid price theory and the problem of multiple recipes \todo{cover this with sources} with more fitting ones or more modern ones. 
	
	It is also clear that its application can be beneficial not only in a context of increasing profits but also in the goal to construct a healthier economy, thus improving the lives of ordinary people.
	
	\todo{the success of IO analysis of Leontief driven largely by his ambitious will to introduce empiric data to economic science and use the theories on practical useful examples aimed to understanding the inner-workings of a system. This is politically unbiased and later proven to be sought by all (including USSR who resists capitalist inventions at all cost)}
	
	%was oder wen hat Leontief beeinflusst und welche seiner Arbeiten hat fuer Sie eine wichtige Bedeutung, vielleicht bis heute.
	
	Leontief's philosophy of improving economics as something that can be more than just a "rule of thumb" \cite[p.~36]{leontief1960niedergang}.
	One of his main contributions and passions is bringing structure to seemingly chaotic occurrences. 	"Most economists continue to rely on “professional intuition” and “sound judgement” to establish the connection between the facts and the theory of economics" \cite[p.~13]{leontief1951input}
	





	\todo{information about Prize nominations cannot be disclosed publicly for 50 years. https://www.nobelprize.org/nomination/peace/}
	
	
	
	\subsection{subsection etc}
	
	\missingfigure{a picture can be inserted here}
		
	\listoftodos

	\bibliography{../citations/history}
	
\end{document}
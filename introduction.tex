
\section{Introduction} \label{introduction}


Known for his contribution of the Input-Output Analysis, Wassily Leontief (1906 - 1999) has been awarded the Nobel Prize in 1973 honoring "the development of input-output method and for its application to important economic problems." Input-Output Analysis can be used for calculating interdependence of components within an economic production system of any size, from a private company to a whole country. This method has found wide usage across the globe in a multitude of economic fields, both in the western and soviet worlds.

In \ref{life} we look at Leontief from a personal perspective, his educational background and his political tendencies. This will then enable us to put his work in context. The notion of looking at economic accounts as a set of interconnected and interdependent pieces was not a completely new concept when Leontief developed his method, though he certainly improved upon earlier attempts and made them actually useful and actionable. \cite[p.~512]{Baumol2009}. In this paper we will look at the influences that lead Leontief to introduce his work to the world in \ref{influences}, a short overview of how the method works \ref{analysis} and its adoption in USA and beyond \ref{contribution}

The Nobel Memorial Prize in Economic Sciences is awarded to "those who [...] shall have conferred the greatest benefit on mankind." \cite[]{Nobel} in the field of economics. Some 37 years of refinements and expanding usage passed from the time Leontief constructed his first input-output tables in 1936 when he calculated the figures based on actual historical data for "the Economic System of the United States, 1919" \cite[]{kohli2003leontief}. Since then this method has been acclaimed across many countries and industries to improve planning and understand relations withing an economic system. Leontief supported the usage and improvement of the method throughout his career as well as contributed strongly in his effort to "making quantitative data more accessible and more indispensable to economics" \cite[p.75]{dymond2015recent} , thus transforming the nature of economics in general to be more of a formal science rather than a social one.
